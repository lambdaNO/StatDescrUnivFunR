\begin{figure}[H]\begin{center}\includegraphics[scale=1]{ilu/probab .png}\end{center}\end{figure}


\begin{lstlisting}[language=html]
  
\end{lstlisting}


%% La dernière image est probab28









%%%%%%%%%%%%%%%%%%%%%%%%%%%%%%%%%%%%%%%%%%%%%%%%%%%%%%%%%%%%%%%%%%%%%%%%%%
%%%%%%%%%%%%%%%%%%%%%%%%%%%%%%%%%%%%%%%%%%%%%%%%%%%%%%%%%%%%%%%%%%%%%%%%%%%%%%%%%%%%%%%%%%%%%%%%%%%%%%%%%%%%%%%
%%%%%%%%%%%%%%%%%%%%%%%%%%%%%%%%%%%%%%%%%%%%%%%%%%%%%%%%%%%%%%%%%%%%%%%%%%%%%%%%%%%%%%%%%%%%%%%%%%%%%%%%%%%%%%%
%%%%%%%%%%%%%%%%%%%%%%%%%%%%%%%%%%%%%%%%%%%%%%%%%%%%%%%%%%%%%%%%%%%%%%%%%%%%%%%%%%%%%%%%%%%%%%%%%%%%%%%%%%%%%%%%%%%%%%%%%%%%%%%%%%%%%%%%%%%%%%%%%%%%%%%%%%%%%%%%%%%%%%%%%%%%%%%%%%%%%%%%%
%%%%%%%%%%%%%%%%%%%%%%%%%%%%%%%%%%%%%%%%%%%%%%%%%%%%%%%%%%%%%%%%%%%%%%%%%%%%%%%%%%%%%%%%%%%%%%%%%%%%%%%%%%%
%%%%%%%%%%%%%%%%%%%%%%%%%%%%%%%%%%%%%
%%%%%%%%%%%%%%%%%%%%%%%%%%%%%%%%%%%%%%%%%%%%%%%%%%%%%%%%%%%%%%%%%%%%%%%%%%%%%%%%%%%%%%%%%%%%%%%%%%%%%%%%%%%%%%%%%%%%%%%%%%%%%%%%%%%%%%%%%%%%%%%%%%%%%%%%%%%%%%%%%%%%%%%%%%%%%%%%%%%%%%%%%%%%%%%%%%%%%%%%%%%%%%%%%%%%%%%%%%%%%%%%%%%%%%%%%%%%%%%%%%%%%%%%%%%%%%%%%%%%%%%%%%%%%%%%%%%%%%%%%%%%%%%%%%%%%%%%%%%%%%%%%%%%%%%%%%%%%%%%%%%%%%%%%%%%%
%%%%%%%%%%%%%%%%%%%%%%%%%%%%%%%%%%%%%
%%%%%%%%%%%%%%%%%%%%%%%%%%%%%%%%%%%%%
%%%%%%%%%%%%%%%%%%%%%%%%%%%%%%%%%%%%%
%%%%%%%%%%%%%%%%%%%%%%%%%%%%%%%%%%%%%






%%%%%%%%%%%%%%%%%%%%%%%%%%%%%%%%%%%%%
%%%%%%%%%%%%%%%%%%%%%%%%%%%%%%%%%%%%%%%%%%%%%%%%%%%%%%%%%%%%%%%%%%%%%%%%%%
%%%%%%%%%%%%%%%%%%%%%%%%%%%%%%%%%%%%%%%%%%%%%%%%%%%%%%%%%%%%%%%%%%%%%%%%%%
%%%%%%%%%%%%%%%%%%%%%%%%%%%%%%%%%%%%%%%%%%%%%%%%%%%%%%%%%%%%%%%%%%%%%%%%%%
%%%%%%%%%%%%%%%%%%%%%%%%%%%%%%%%%%%%%

\begin{figure}[H]\begin{center}\includegraphics[scale=1]{ilu/adeq .png}\end{center}\end{figure}



Soit $X$ la \textit{var} égale au numéro affiché par le dé après un lancer. Par l'énoncé, on observe la valeur de $X$ sur chacun des $n$ individus (dés) d'un échantillon avec $n = 100$ : $(x_{1}; \dots ; x_{n})$ (avec $x_{i} \in \{1; \dots ; 6\})$. On forme alors un vecteur des effectifs $(n_{1}; n_{2}; \dots ; n_{6}) = (18; 23; \dots ; 15)$.\newline
Dire que le dé n'est pas truqué signifie que $X$ suit la loi uniforme $\mathit{U}(\{1; \dots ; 6\})$ :
$$\mathbb{P}(X=i) = \frac{1}{6}, \textrm{ } i \in \{1,\dots, 6\}$$
La problématique est la suivante : \textit{est-ce que les données nous permettent d'affirmer que X ne suit pas la loi $\mathit{U}(\{1; \dots ; 6\})$} ?\newline
\\
Ainsi, pour une analyse graphique, on propose les commandes :
\begin{lstlisting}[language=html]
> nb = c(18, 23, 19, 12, 11, 15)
> bar = barplot(nb / 100, col = "white")
> points(bar, rep(1 / 6, 6), type = "h",col="red")
\end{lstlisting}
\begin{figure}[H]\begin{center}\includegraphics[scale=1]{ilu/adeq14.png}\end{center}\end{figure}
Il est difficile de conclure au vu des différences observées.\newline
On peut utiliser le test du Chi-deux d'adéquation à une loi pour y voir plus clair. On considère alors les hypothèses :
\begin{itemize}
  \item $H_{0}$ : "X suit la loi uniforme $\mathit{U}(\{1; \dots ; 6\})$"
  \item $H_{1}$ : "X ne suit pas la loi uniforme $\mathit{U}(\{1; \dots ; 6\})$".
\end{itemize}
Les valeurs sont regroupées en $k = 6$ classes : $C_{1} = \{1\}$, $C_{2} = \{2\}$,\dots, $C_{6} = \{6\}$.\newline
On considère les commandes :
\begin{lstlisting}[language=html]
> nb
[1] 18 23 19 12 11 15
> proba = rep(1 / 6, 6)
> chisq.test(nb, p = proba)$p.value
[1] 0.2757299
\end{lstlisting}
Notons qu'aucun "Warning message" n'apparaît ; les conditions d'application du test sont vérifiées.\newline
Comme $p$-valeur $> 0,05$, les données ne nous permettent pas de rejeter $H_{0}$. Ainsi, on ne peut pas affirmer que le dé est truqué.

%%%%%%%%%%%%%%%%%%%%%%%%%%%%%%%%%%%%%

%%%%%%%%%%%%%%%%%%%%%%%%%%%%%%%%%%%%%

On indique dans le tableau suivant le nombre de fois qu'un chiffre apparaît dans les $608$ premières décimales de $\pi$ : 
\begin{center}
\begin{tabular}{|c|c|c|c|c|c|c|c|c|c|c|}
\hline
\textbf{Chiffre}        & 0  & 1  & 2  & 3  & 4  & 5  & 6  & 7  & 8  & 9  \\ \hline
\textbf{Nombre de fois} & 60 & 62 & 67 & 68 & 64 & 56 & 62 & 44 & 58 & 67 \\ \hline
\end{tabular}
\end{center}
\textit{Peut-on affirmer, au risque $5\%$, que les décimales ne sont pas équiréparties ?}

%%%%%%%%%%%%%%%%%%%%%%%%%%%%%%%%%%%%%

%%%%%%%%%%%%%%%%%%%%%%%%%%%%%%%%%%%%%

Soit $X$ la var égale au chiffre affiché par une décimale (que l'on suppose donc aléatoire).\newline
Par l'énoncé, on observe la valeur de $X$ sur chacun des n individus (décimales) d'un échantillon avec $n = 608$ : $(x_{1}; \dots ; x_{n})$ (avec $x_{i} \in \{0; \dots ; 9 \})$. On forme alors un vecteur des effectifs $(n_{1}; n_{2}; \dots ; n_{10}) = (60; 62; : : : ; 67)$. Dire que les décimales sont équiréparties signifie que $X$ suit la loi uniforme $\mathit{U}(\{0; \dots ; 9\})$ :
$$\mathbb{P}(X=i) = \frac{1}{10}, \textrm{ } i \in \{1,\dots, 9\}$$

La problématique est la suivante : \textit{est-ce que les données nous permettent d'affirmer que X ne suit pas la loi $\mathit{U}(\{0; \dots ; 9\})$} ?\newline
\\
\begin{itemize}
  \item $H_{0}$ : "X suit la loi uniforme $\mathit{U}(\{1; \dots ; 9\})$"
  \item $H_{1}$ : "X ne suit pas la loi uniforme $\mathit{U}(\{1; \dots ; 9\})$".
\end{itemize}
On peut utiliser le test du Chi-deux d'adéquation à une loi. Les valeurs sont regroupées en k = 10
classes : $C_{1} = \{0\}$, $C_{2} = \{1\}$,\dots, $C_{10} = \{9\}$.\newline
On considère les commandes :
\begin{lstlisting}[language=html]
> nb = c(60, 62, 67, 68, 64, 56, 62, 44, 58, 67)
> proba = rep(1 / 10, 10)
> chisq.test(nb, p = proba)$p.value
[1] 0.585888
\end{lstlisting}
Notons qu'aucun "Warning message" n'apparaît ; les conditions d'application du test sont vérifiées.\newline
Comme $p$-valeur $> 0,05$, les données ne nous permettent pas de rejeter $H_{0}$. Ainsi, au risque $5\%$, on ne peut pas affirmer que le décimales de $\pi$ soient équiréparties.

%%%%%%%%%%%%%%%%%%%%%%%%%%%%%%%%%%%%%%%%%%%%%
%%%%%%%%%%%%%%%%%%%%%%%%%%%%%%%%%%%%%%%%%%%%%
%%%%%%%%%%%%%%%%%%%%%%%%%%%%%%%%%%%%%%%%%%%%%
%%%%%%%%%%%%%%%%%%%%%%%%%%%%%%%%%%%%%%%%%%%%%
%%%%%%%%%%%%%%%%%%%%%%%%%%%%%%%%%%%%%%%%%%%%%
%%%%%%%%%%%%%%%%%%%%%%%%%%%%%%%%%%%%%%%%%%%%%
%%%%%%%%%%%%%%%%%%%%%%%%%%%%%%%%%%%%%%%%%%%%%
%%%%%%%%%%%%%%%%%%%%%%%%%%%%%%%%%%%%%%%%%%%%%
%%%%%%%%%%%%%%%%%%%%%%%%%%%%%%%%%%%%%%%%%%%%%
%%%%%%%%%%%%%%%%%%%%%%%%%%%%%%%%%%%%%%%%%%%%%

Lindström, spécialiste de la génétique et de l'hybridation du maïs, a croisé deux types récessifs de maïs : le type vert-zébré et le type doré. Si les lois de la génétique sont respectées obtient :
\begin{itemize}
  \item  "vert" avec la probabilité $9/16$
  \item  "doré" avec la probabilité $3/16$
  \item  "vert-zébré" avec la probabilité $3/16$
  \item  "doré-vert-zébré" avec la probabilité $1/16$
\end{itemize}
On effectue $1301$ croisements. On obtient les résultats suivants :
\begin{center}
\begin{tabular}{|c|c|c|c|c|}
\hline
\textbf{Type}          & "vert" & "doré" & "vert-zébré" & "doré-vert-zébré" \\ \hline
\textbf{Nombre defois} & 773    & 231     & 238            & 59                   \\ \hline
\end{tabular}
\end{center}
\textit{Peut-on dire que les lois de la génétique ne sont pas respectées ?} (on fera une analyse graphique convenable, puis un test statistique adapté au risque $5\%$).

\varepsilon

%%%%%%%%%%%%%%%%%%%%%%%%%%%%%%%%%%%%%%%%%%%%%
%%%%%%%%%%%%%%%%%%%%%%%%%%%%%%%%%%%%%%%%%%%%%
%%%%%%%%%%%%%%%%%%%%%%%%%%%%%%%%%%%%%%%%%%%%%
%%%%%%%%%%%%%%%%%%%%%%%%%%%%%%%%%%%%%%%%%%%%%
%%%%%%%%%%%%%%%%%%%%%%%%%%%%%%%%%%%%%%%%%%%%%
%%%%%%%%%%%%%%%%%%%%%%%%%%%%%%%%%%%%%%%%%%%%%
%%%%%%%%%%%%%%%%%%%%%%%%%%%%%%%%%%%%%%%%%%%%%
%%%%%%%%%%%%%%%%%%%%%%%%%%%%%%%%%%%%%%%%%%%%%
%%%%%%%%%%%%%%%%%%%%%%%%%%%%%%%%%%%%%%%%%%%%%
%%%%%%%%%%%%%%%%%%%%%%%%%%%%%%%%%%%%%%%%%%%%%













\begin{figure}[H]\begin{center}\includegraphics[scale=1]{ilu/norm .png}\end{center}\end{figure}