\begin{description}
\item[return(value)] mis dans expr lors d'une d�finition de fonction, indique que la fonction doit renvoyer ce r�sultat(si return est absent, la fonction renvoie la derni�re valeur calcul�e dans expr)

\item[if(cond) \{expr\}] si cond est vrai (TRUE), �valuer expr

\item[== != < > <= >=] op�rateurs de comparaison, dans l'ordre: �gal, diff�rent, inf�rieur, sup�rieur, inf�rieur ou �gal, sup�rieur ou �gal; \textit{e.g.} 1==1 vaut TRUE ou T; 1!=1 vaut FALSE ou F; dans les op�rations avec des nombres, T est converti en 1 et F en 0 (T-1==0 est vrai)

\item[if(cond) \{cons.expr\} else \{alt.expr\}] si cond est vrai �valuer cons.expr sinon �valuer alt.expr

\item[for(var in seq) \{expr\}] ex�cute l'expression pour chaque valeur de var prises dans une sequence

\item[while(cond) \{expr\}] ex�cute l'expression tant que la condition est vraie

\item[repeat \{expr\}] r�p�te expr en boucle; penser � l'arr�ter avec if(...) \{break\} (ou avec les touches Ctrl+C)

\item[break] arr�te une boucle for, while ou repeat

\item[next] arr�te l'it�ration en cours et reprend la boucle (dans le cas de for, avec la valeur suivante de la sequence)

\item[ifelse(test, yes, no)] pour chaque ligne/cellule de test, renvoie la valeur yes si le test
\end{description}