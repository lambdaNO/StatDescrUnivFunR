

\section*{Exercices}
\subsection*{Enoncés}


\begin{lstlisting}[language=html]
> fonc1 = function(m, n){
+   x = 1:n
+   y = x^m
+   sum(y)
+ }
> fonc1(10,2)
[1] 1025
\end{lstlisting}


\begin{lstlisting}[language=html]

\end{lstlisting}


\begin{figure}[H]\begin{center}\includegraphics[scale=0.4 ]{ilu/.png}\end{center}\end{figure}


%% remplacer les ''''

\subsubsection*{Exercice 5}
Créer dans R la fonction $f : \mathbb{R}^{n} \times \mathbb{R}^{n}$ définie par :
$$f(m,n)=\sum_{k=1}^{n} k^{m}$$



\subsubsection*{Exercice 7}
Créer dans R la fonction $f : \mathbb{R}^{n} \times \mathbb{R}^{n}$ définie par :
$$f(x_{1},\dots,x_{n}) = \left(x_{1},\frac{x_{2}^{2}}{2},\dots,\frac{x_{n}^{n}}{n}\right)$$

\subsubsection*{Exercice 9}
Créer dans R la fonction $f : \mathbb{N}^{2} \times [0;1] \rightarrow [0; 1]$ définie par :
$$f(k,n,p) = \binom{n}{k}p^{k}(1-p)^{n-k}$$